The work with the thesis will begin in August 2012 and will be completed by May 2013.


The following tools and resources will be used to complete the project:

\begin{description}
    \item[Language]: Java \footnote{\url{http://www.java.com/en/}} 
    \\ Java is a high level object oriented language, with good options for creating web applications.
    \item[Framework]: Spring \footnote{\url{http://www.springsource.org/}} 
    \\ Spring will be used as a development framework. Spring is a common framework in enterprise development, which helps in creating high level code.
    \item[Integrated Development Environment]: The Springsource Tool Suite(STS)\footnote{\url{http://www.springsource.com/developer/sts}} 
    \\ STS will be used as an Integrated Development Environment.
    \item[Server]: Jetty\footnote{\url{http://www.eclipse.org/jetty/}} and Tomcat\footnote{\url{http://tomcat.apache.org/}} 
    \\ Jetty is a light weight web server which will be used through development to test the web application. When the project is to be deployed for testing and use we will use a Tomcat server for additional robustness.
    \item[Version Control]: Git\footnote{\url{http://git-scm.com/}} and GitHub\footnote{\url{https://github.com/}} 
    \\ Git is a open source distributed version control system. Using this will make it easier for others to see and continue the development of \theartefact after the thesis is completed. The git repository will also be hosted on GitHub. This will both give the project an external backup strategy, and make it easier to share the results when the project is over.
    \item[Typesetting]: \LaTeX{}\footnote{\url{http://www.latex-project.org/}}
    \\ The thesis will be written using \LaTeX{}. This will make it easier to separate the text from its formatting, keep proper references and citations, and if necessary change citation style.
\end{description}

\begin{ganttchart}{12}
  \gantttitle{2012}{12} \\
  \gantttitlelist{1,...,12}{1} \\
  \ganttgroup{Group 1}{1}{7} \\
  \ganttbar{Task 1}{1}{2} \\
  \ganttlinkedbar{Task 2}{3}{7} \ganttnewline
  \ganttmilestone{Milestone}{7} \ganttnewline
  \ganttbar{Final Task}{8}{12}
  \ganttlink{elem2}{elem3}
  \ganttlink{elem3}{elem4}
\end{ganttchart}